\documentclass{article}  
\usepackage[margin=3cm]{geometry}
\usepackage{hyperref}

\begin{document}  
  
\begin{center}
    {\Large {\bf Precision weeding in lettuce and spinach fields}} \\
    \vspace{0.3cm}
    {\Large Literature Review} \\
    
    \vspace{1cm}
\end{center}

{\large {\bf Abstract}}

In this research project, we are trying to develop a method for Precision weeding in crop fields. Our initial step is to find the best approach to create a heatmap of the crops and weeds on the whole filed. Once we have such a map, we can feed it to a machine that is capable of spraying areas marked as weed in the field. There have been various approaches in detecting weeds in a crop filed. We will try to summarize many of those useful approaches here. \\

\section{Unsupervised Classification Algorithm for Early
Weed Detection in Row-Crops by Combining
Spatial and Spectral Information ~\cite{louargant-2018-mdpi-unsupervised}}
	
	This paper introduces an automatic approach in weed detection in order to reduce herbicide use in agriculture. They combine spatial and spectral information/features from four band multi spectral images. Images acquired using a camera mounted on a pole 3m above the ground. The automatic part of this method is as follows. They consider row crop arrangement as a ground truth in grouping the plants. In other words, everything not on the crop rows will be considered weed and things on the crop row might be either weed or plant. Then using these informations, a training dataset containing the multi-spectral image features is created and use to train a support vector machine (SVM). At the end for testing, inter-row crops are classified as weed (only based on spatial feature) and in-row crops are classified as either weed or crop using the SVM and the spectral features. They report 89 as their weed detection rate using the method mentioned earlier (combination of both spatial and spectral features), 79 for using only spatial features and 75 for using only spectral features. 
	
	\subsection{More on this method}
	
	One way to distinguish different approaches for detection and localization of weeds is the distance and the vehicle from which the images are captured which includes satellite, aerial, terrestrial vehicle, etc. Another one is the sensor used to capture images. Images can only have the three color bands (RGB) or might have other bands as well such as Near Infrared. One interesting approach is to use UAVs to capture multi-spectral images from large areas in a field. Based on the previous studies, weeds and crops can be discriminated using their reflectance spectra. However, there are some limitations: number of spectral bands is limited, field condition affects the spectral reflectance information, spectral reflectance changes with psychological stress. 
	
	At the time that this paper was written, the usual camera for weed detection on a UAV had 4 bands (near infrared + RGB). NIR helps separation of the vegetation and the background. 
	
	\subsection{Vegetation Indices}
	
	In most of the weed detection method, the first step is to distinguish the vegetation (whether it is weed or crop) from the background. It is usually done by calculating a Vegetation Index and applying a thresholding algorithm on that index value at each pixel. One of the most common indices is Normalized Difference Vegetation Index (NDVI) for the case where NIR data is available and Excess Green Index (ExG) when only RGB is available. 
	
	\subsection{Geometrical Information for weed detection}
	
	One approach for weed detection is to use geometrical information for detecting weeds. One is automatic row detection using textural based methods. People have used Gabor filters and Object-Based Image Analysis (OBIA) to detect the crop rows by segmentation. OBIA is something like a histogram of texture. 
	
	\subsection{Some Spatial and Spectral combination methods}
	
	There has been some methods that combined spatial and spectral features together using Bayesian methods and pixel neighborhood similarity. Another interesting method was using spatial information to detect inter-row weeds and then using some sort of nearest neighbor classifier with the vegetation index to classify in-row crops. Another method combined SVM and OBIA and reached 96 percent accuracy in sunflower fields. They also added a feature selection method which helped the accuracy. In some other methods, hough transform and randome forest are used which reached 96 percent accuracy. Another paper reconstructed the plants in 3D and used their average height to decide whether they are weed or crop. 
	
	In general, supervised classification is much better than unsupervised but it needs labeled training data. On the other hand, unsupervised methods can be used to generated labeled data. 
	
\bibliographystyle{plain}
\bibliography{Agriculture_Literature_Review}
	
\end{document}